\documentclass[11pt,a4paper]{article}
\usepackage[latin1,utf8]{inputenc}
\usepackage{natbib}
\usepackage[english, french]{babel}
\usepackage{amsmath}
\usepackage{amsfonts}
\usepackage{amssymb}
\usepackage{graphicx}
\usepackage{colortbl}
\usepackage{multirow}
\usepackage{xcolor}
\usepackage{geometry}
\usepackage{subfig}
\usepackage{tikz}
\maxdeadcycles=1000
%\geometry{ left = 60pt , right = 60pt }
\graphicspath{{./Fig/}}

\definecolor{mypurple}{RGB}{127,0,255}
\newcommand{\CA}[1]{{\color{mypurple} \emph{CA : #1} }}

\begin{document}
\paragraph{First model considered for adipocytes ``growing''.\\}
$T$ triglycerides (= lipids)  in the blood\\
$u(t,r)$ adipocytes number at time $t$ of radius $r$ (containing $l$ lipids)  \\
The link between $r$ and $l$ is the following,
$$V_l l = \dfrac{4}{3} \pi r^3 - V_{0}$$
with $V_{0} = \dfrac{4}{3} \pi r_{0}^3$  the minimum volume of  adipocytes ($r_{0}$ is the minimum radius), so $\dfrac{dl}{dt} = \dfrac{4\pi}{V_{l}} \dfrac{dr}{dt} r^2$.\\

Model the lipogenesis and the lipolysis (fluxes of triglycerides $l$) : 
$$\dfrac{dl}{dt} = a r^2 \dfrac{T}{T + T_{\theta}} \dfrac{r_{\theta}^n}{r^n+r_{\theta}^n} - (B + br^2) \dfrac{l}{l + \ell_{\theta}}$$
we rewrite with the variable $r$ : 
$$\dfrac{dr}{dt} = a\dfrac{V_{l}}{4\pi} \dfrac{T}{T + T_{\theta}} \dfrac{r_{\theta}^n}{r^n+r_{\theta}^n}  - \dfrac{V_{l}}{4\pi}\dfrac{(B + br^2)}{r^2} \dfrac{\dfrac{4}{3} \pi r^3 - V_{0}}{\dfrac{4}{3} \pi r^3 - V_{0} +V_{l} \ell_{\theta}} = \tau(r, T)$$
From these fluxes, the dynamics of the number of adypocytes $u$ is described by 
$$\partial_t u(t, r) + \partial_r( \tau(r, T) u - D \partial_r u) = 0$$
with $D$ a diffusion coefficient.

The intracellular quantity of triglycerides is $U(t) = \displaystyle \int l \rho_u dl$ with $\rho_u$ adipocytes density. The total amount of triglycerides is assumed constant over time : $\dfrac{d}{dt}(T(t) + U(t)) = 0$ and $\dfrac{dT}{dt} = -\dfrac{dU}{dt}$ and we denote by $Tg_{0}$ the initial quantity of triglycerides so that $T(t) + U(t)=Tg_{0}$ for every time $t$.
 
\paragraph{boundary/initial conditions.}
$u(0, r) =$ Gaussian density (minimum = $r_{0}$), first test : unimodal (Q: can we recover the bimodal distribution that is observed).\\
$T(t) = Tg_{0}- U(t)$\\
$(\tau(r, T) u(t,r) - D \partial_r u(t,r)) |_{r_{max}}=0$
\paragraph{Recruitment of new adipocytes.} We assumed that when the level of triglycerides increases too largely, pre-adipocytes differentiate into adipocytes. This recruitment is modeled with the $r_{0}$ BC, as follows,
$$(\tau(r, T) u(t,r) - D \partial_r u(t,r)) |_{r_{0}}= f(T)$$ with $f(T) = \alpha T$ or $f(T) = \alpha \dfrac{T}{(T+\kappa_{\theta})}$ 


\end{document}