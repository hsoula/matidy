\documentclass[11pt,a4paper]{article}
\usepackage[latin1,utf8]{inputenc}
\usepackage{natbib}
\usepackage[english]{babel}
\usepackage{amsmath}
\usepackage{amsfonts}
\usepackage{amssymb}
\usepackage{graphicx}
\usepackage{colortbl}
\usepackage{multirow}
\usepackage{xcolor}
\usepackage{geometry}
\usepackage{subfig}
\usepackage{tikz}
\maxdeadcycles=1000
%\geometry{ left = 60pt , right = 60pt }
\graphicspath{{./Fig/}}

\definecolor{mypurple}{RGB}{127,0,255}
\newcommand{\CA}[1]{{\color{mypurple} \emph{CA : #1} }}

\begin{document}
\begin{center}
\large \textbf{Sensitivity analysis}
\end{center}

We consider the stationary problem. 
$$T = T_{g0} - \displaystyle  \int_{r_0}^{r_{max}} \dfrac{V(r) - V_0}{V_l} u(r) \dfrac{4\pi r^2}{V_l} dr$$
$$u(r) = \displaystyle \dfrac{M}{\displaystyle \int_{r_{0}}^{r_{max}} \exp  \left(\displaystyle  \int_{r_{0}}^{r} \dfrac{1}{D} \tau(s, T) ds \right) dr}  \, \exp \left(\displaystyle  \int_{r_{0}}^{r} \dfrac{1}{D} \tau(s, T) ds \right).$$

with $V(r) = \dfrac{4}{3} \pi r^3$ and where $\tau(r, T)$ is defined by 
$$\tau(r, T) = \dfrac{aV_l}{4\pi}  \dfrac{T}{T + T_{\theta}} \dfrac{r_{\theta}^{n_r}}{r^{n_r}+ r_{\theta}^{n_r}} - V_l \dfrac{(B + br^2)}{4\pi r^2} \dfrac{V(r) - V_0}{V(r) - V_0 + V_l  l_{\theta}}.$$
The parameters for which we want to study sensitivity of the model are $(a, T_{\theta}, r_{\theta}, l_{\theta}, D, T_{g0}, M)$. 

We write our problem in the general form
$$F(t, y, \dot{y}, p) = 0$$
where $p$ is the vector of parameters, $y$ the state variables, $\dot{y}$ the time derivative of state variables and $t$ the time. The traditional sensitivity functions are the model variable derivatives with respect to the parameters \citep{thomaseth.1999},
$$s_i(t) = \dfrac{\partial y}{\partial p_i} (t)$$
The functions $s_i$ verify the equations, 
$$\partial_y F \, s_i + \partial_{\dot{y}} F \, \dot{s}_i + \partial_{p_i} F = 0$$
To compute the traditional sensitivity, we are solving the system $F(t, y, \dot{y}, p) = 0$ and $\partial_y F \, s_i + \partial_{\dot{y}} F \, \dot{s}_i + \partial_{p_i} F = 0$, we are computing $y$ and $s$ at the same time. 

We explicitly write the system of equation for our case. 
$$F(r, y, p) = \begin{pmatrix}
y_1 - T_{g0} + \displaystyle  \int_{r_0}^{r_{max}} \dfrac{V(r) - V_0}{V_l} y_2 \dfrac{4\pi r^2}{V_l} dr \\
y_2 - \displaystyle \dfrac{M}{\displaystyle \int_{r_{0}}^{r_{max}} \exp  \left(\displaystyle  \int_{r_{0}}^{r} \dfrac{1}{D} \tau(s, y_1) ds \right) dr}  \, \exp \left(\displaystyle  \int_{r_{0}}^{r} \dfrac{1}{D} \tau(s, y_1) ds \right)
\end{pmatrix} $$
with $y = (y_1, y_2)^{t} = (T, u)^{t}$, and $p =(a, T_{\theta}, r_{\theta}, l_{\theta}, D, T_{g0}, M)^t$.
For the sensitivity equations, $\partial_{\dot{y}} F = 0$ and  
$$\partial_y F = \begin{pmatrix}
\partial_{y_1} F_1 & \partial_{y_2} F_1 \\
\partial_{y_1} F_2 & \partial_{y_2} F_2 \\
\end{pmatrix}$$
with 
$\partial_{y_1} F_1 = 1 $, $\partial_{y_2} F_2 = 1$, 
$$\begin{array}{rcl}
\partial_{y_2} F_1 & = &\displaystyle \partial_{y_2} \left( \int_{r_0}^{r_{max}} \dfrac{V(r) - V_0}{V_l} y_2 \dfrac{4\pi r^2}{V_l} dr \right) \\
& = & \displaystyle \int_{r_0}^{r_{max}} \partial_{y_2} ( \dfrac{V(r) - V_0}{V_l} y_2 \dfrac{4\pi r^2}{V_l} ) dr\\
& = & \displaystyle \int_{r_0}^{r_{max}} \dfrac{V(r) - V_0}{V_l} \dfrac{4\pi r^2}{V_l} dr\\
& = & \displaystyle \int_{r_0}^{r_{max}} \dfrac{(4/3)\pi r^3 - V_0}{V_l} \dfrac{4\pi r^2}{V_l} dr\\
& = & \displaystyle \dfrac{4\pi}{V_l^2} \int_{r_0}^{r_{max}} (4/3)\pi r^5 - V_0 r^2 dr\\
& = & \dfrac{4\pi}{V_l^2} \left( (4/3)\pi \dfrac{r_{max}^6 - r_0^6}{6} - V_0 \dfrac{r_{max}^3 - r_0^3}{3} \right)\\
\end{array}$$
and 
$\partial_{y_1} F_2  = \displaystyle \partial_{y_1} \left[  - \displaystyle \dfrac{M}{\displaystyle \int_{r_{0}}^{r_{max}} \exp  \left(\displaystyle  \int_{r_{0}}^{r} \dfrac{1}{D} \tau(s, y_1) ds \right) dr}  \, \exp \left(\displaystyle  \int_{r_{0}}^{r} \dfrac{1}{D} \tau(s, y_1) ds \right) \right] $, let set $g(r, y_1) = \displaystyle  \int_{r_{0}}^{r} \dfrac{1}{D} \tau(s, y_1) ds$, so 
$$\begin{array}{rcl}
\partial_{y_1} F_2 & = & \displaystyle \partial_{y_1} \left[  - \dfrac{M}{\displaystyle \int_{r_{0}}^{r_{max}} \exp  (g(r, y_1)) dr}  \, \exp \left( g(r, y_1) \right) \right]  \\

 & = &\displaystyle -M \dfrac{ (\partial_{y_1}g) \exp(g)  \int_{r_{0}}^{r_{max}} \exp (g) dr - \exp(g) \partial_{y_1} \left(\int_{r_{0}}^{r_{max}} \exp (g) dr\right)}{\left( \int_{r_{0}}^{r_{max}} \exp (g) dr \right)^2} \\

 & = &\displaystyle -M \exp(g) \dfrac{ (\partial_{y_1}g) \int_{r_{0}}^{r_{max}} \exp (g) dr - \int_{r_{0}}^{r_{max}} (\partial_{y_1}g) \exp (g) dr }{\left( \int_{r_{0}}^{r_{max}} \exp (g) dr \right)^2} \\
 \end{array}$$
and 
$$\partial_{y_1}g =\displaystyle \partial_{y_1} \left( \int_{r_{0}}^{r} \dfrac{1}{D} \tau(s, y_1) ds \right) = \int_{r_{0}}^{r} \dfrac{1}{D} \partial_{y_1}\tau(s, y_1) ds $$
with
$$ \partial_{y_1}\tau(s, y_1) = \dfrac{aV_l}{4\pi}  \dfrac{T_{\theta}}{(y_1 + T_{\theta})^2} \dfrac{r_{\theta}^{n_r}}{s^{n_r}+ r_{\theta}^{n_r}}$$

\noindent
The final term to clarify is $\partial_{p_i} F$ for each $p_i$ in $(a, T_{\theta}, r_{\theta}, l_{\theta}, D, T_{g0}, M)$. 

\noindent
For $ p_0 = a $, we get, 
$$\partial_{p_0} F = \begin{pmatrix} 
0 \\ 
\displaystyle -M \exp(g) \dfrac{ (\partial_{p_0}g) \int_{r_{0}}^{r_{max}} \exp (g) dr - \int_{r_{0}}^{r_{max}} (\partial_{p_0}g) \exp (g) dr }{\left( \int_{r_{0}}^{r_{max}} \exp (g) dr \right)^2}
\end{pmatrix} 
$$
with $$\begin{array}{rcl}
\partial_{p_0}g & = & \displaystyle \partial_{p_0} \left( \int_{r_{0}}^{r} \dfrac{1}{D} \tau(s, y_1) ds \right)\\
& = & \displaystyle \int_{r_{0}}^{r} \dfrac{1}{D} \partial_{p_0}\tau(s, y_1) ds \\
& = & \displaystyle \int_{r_{0}}^{r} \dfrac{1}{D} \dfrac{V_l}{4\pi}  \dfrac{y_1}{y_1 + T_{\theta}} \dfrac{r_{\theta}^{n_r}}{s^{n_r}+ r_{\theta}^{n_r}} ds \\
& = & \dfrac{1}{D} \dfrac{V_l}{4\pi}  \dfrac{y_1}{y_1 + T_{\theta}} \displaystyle \int_{r_{0}}^{r}  \dfrac{r_{\theta}^{n_r}}{s^{n_r}+ r_{\theta}^{n_r}} ds
\end{array}$$

\noindent
For $ p_1 = T_{\theta} $, we get, 
$$\partial_{p_1} F = \begin{pmatrix} 
0 \\ 
\displaystyle -M \exp(g) \dfrac{ (\partial_{p_1}g) \int_{r_{0}}^{r_{max}} \exp (g) dr - \int_{r_{0}}^{r_{max}} (\partial_{p_1}g) \exp (g) dr }{\left( \int_{r_{0}}^{r_{max}} \exp (g) dr \right)^2}
\end{pmatrix} 
$$
with $$\begin{array}{rcl}
\partial_{p_1}g & = & \displaystyle \partial_{p_1} \left( \int_{r_{0}}^{r} \dfrac{1}{D} \tau(s, y_1) ds \right)\\
& = & \displaystyle \int_{r_{0}}^{r} \dfrac{1}{D} \partial_{p_1}\tau(s, y_1) ds \\
& = & \displaystyle \int_{r_{0}}^{r} \dfrac{1}{D} \dfrac{aV_l}{4\pi}  \dfrac{-y_1}{(y_1 + T_{\theta})^2} \dfrac{r_{\theta}^{n_r}}{s^{n_r}+ r_{\theta}^{n_r}} ds  \\
& = & \dfrac{1}{D} \dfrac{aV_l}{4\pi}  \dfrac{-y_1}{(y_1 + T_{\theta})^2} \displaystyle \int_{r_{0}}^{r}  \dfrac{r_{\theta}^{n_r}}{s^{n_r}+ r_{\theta}^{n_r}} ds 
\end{array}$$


\noindent
For $ p_2 = r_{\theta} $, we get, 
$$\partial_{p_2} F = \begin{pmatrix} 
0 \\ 
\displaystyle -M \exp(g) \dfrac{ (\partial_{p_2}g) \int_{r_{0}}^{r_{max}} \exp (g) dr - \int_{r_{0}}^{r_{max}} (\partial_{p_2}g) \exp (g) dr }{\left( \int_{r_{0}}^{r_{max}} \exp (g) dr \right)^2}
\end{pmatrix} 
$$
with $$ \partial_{p_2}g  = \dfrac{1}{D} \dfrac{aV_l}{4\pi}  \dfrac{y_1}{y_1 + T_{\theta}} \displaystyle \int_{r_{0}}^{r} \dfrac{n_r s^{n_r} r_{\theta}^{n_r - 1}}{(s^{n_r}+ r_{\theta}^{n_r})^2} ds $$



\noindent
For $ p_3 = l_{\theta} $, we get, 
$$\partial_{p_3} F = \begin{pmatrix} 
0 \\ 
\displaystyle -M \exp(g) \dfrac{ (\partial_{p_3}g) \int_{r_{0}}^{r_{max}} \exp (g) dr - \int_{r_{0}}^{r_{max}} (\partial_{p_3}g) \exp (g) dr }{\left( \int_{r_{0}}^{r_{max}} \exp (g) dr \right)^2}
\end{pmatrix} 
$$
with $$ \partial_{p_3}g  =  \displaystyle \int_{r_{0}}^{r} \dfrac{1}{D} V_l \dfrac{(B + bs^2)}{4\pi s^2} \dfrac{-(V(s) - V_0)V_l}{(V(r) - V_0 + V_l  l_{\theta})^2} ds $$


\noindent
For $ p_4 = D $, we get, 
$$\partial_{p_4} F = \begin{pmatrix} 
0 \\ 
\displaystyle -M \exp(g) \dfrac{ (\partial_{p_4}g) \int_{r_{0}}^{r_{max}} \exp (g) dr - \int_{r_{0}}^{r_{max}} (\partial_{p_4}g) \exp (g) dr }{\left( \int_{r_{0}}^{r_{max}} \exp (g) dr \right)^2}
\end{pmatrix} 
$$
with $$ \partial_{p_4}g  =  \displaystyle \int_{r_{0}}^{r} \dfrac{-1}{D^2} \tau(s, y_1) ds $$


\noindent
For $ p_5 = T_{g0} $, we get, 
$$\partial_{p_5} F = \begin{pmatrix} 
-1 \\  0
\end{pmatrix} 
$$


\noindent
For $ p_6 = M $, we get, 
$$\partial_{p_6} F = \begin{pmatrix} 
0 \\ 
- \dfrac{1}{\displaystyle \int_{r_{0}}^{r_{max}} \exp  \left(\displaystyle \int_{r_{0}}^{r} \dfrac{1}{D} \tau(s, y_1) ds \right) dr}  \, \exp \left(\displaystyle  \int_{r_{0}}^{r} \dfrac{1}{D} \tau(s, y_1) ds \right)
\end{pmatrix} 
$$


\vspace{1cm}
In our Python code, we implement the resolution of the system, 
$$\begin{array}{l}
F(r, y, p) = 0 \\
\partial_y F \, s_i + \partial_{p_i} F = 0 , \, \text{for each }p_i, s_i, i \in \lbrace 0, \dots, 6\rbrace
\end{array}$$

\vspace{1cm}


\begin{thebibliography}{10}

\bibitem[Thomaseth and Cobelli(1999)]{thomaseth.1999} Thomaseth K. and Cobelli C. (1999) Generalized Sensitivity Functions in Physiological System Identification. Annals of Biomedical Engineering, Vol. 27, 607--616.

\end{thebibliography}

\end{document}