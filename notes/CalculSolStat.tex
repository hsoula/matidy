\documentclass[11pt,a4paper]{article}
\usepackage[latin1,utf8]{inputenc}
\usepackage{natbib}
\usepackage[english, french]{babel}
\usepackage{amsmath}
\usepackage{amsfonts}
\usepackage{amssymb}
\usepackage{graphicx}
\usepackage{colortbl}
\usepackage{multirow}
\usepackage{xcolor}
\usepackage{geometry}
\usepackage{subfig}
\usepackage{tikz}
\maxdeadcycles=1000
%\geometry{ left = 60pt , right = 60pt }
\graphicspath{{./Fig/}}

\definecolor{mypurple}{RGB}{127,0,255}
\newcommand{\CA}[1]{{\color{mypurple} \emph{CA : #1} }}

\newcommand{\dint}{\displaystyle \int}

\begin{document}
\begin{center}
\large \textbf{Stationary solution(s)}
\end{center}
\CA{update may 2020}

Search for stationary solution of our first model (see \textit{FirstModelAssumptions.pdf}), $u^{\infty}(r)$, that verifies $\partial_t u^{\infty}(t, r) = 0$ so the following equation
$$\partial_t u(t, r) + \partial_r( \tau(r, T(t))\, u(t, r) - D \partial_r u(t,r)) = 0$$
$$T(t) = T_{g0} - \int_{r_0}^{r_{max}} \dfrac{V(r) - V_0}{V_l} u(t, r) \dfrac{4\pi r^2}{V_l} dr$$
with $V(r) = \dfrac{4}{3} \pi r^3$ and where $\tau(r, T)$ is defined by 
$$\tau(r, T) = \dfrac{aV_l}{4\pi}  \dfrac{T}{T + T_{\theta}} \dfrac{r_{\theta}^{n_r}}{r^{n_r}+ r_{\theta}^{n_r}} - V_l \dfrac{(B + br^2)}{4\pi r^2} \dfrac{V(r) - V_0}{V(r) - V_0 + V_l  l_{\theta}}.$$
with the boundary conditions:
$$(\tau(r, T) u(t,r) - D \partial_r u(t,r)) |_{r_{0}} = 0$$
$$(\tau(r, T) u(t,r) - D \partial_r u(t,r)) |_{r_{max}} = 0$$
A consequence is that number of cells is constant, for all $t$, $$ \displaystyle \int_{r_0}^{r_{max}} u(t, r) dr = \displaystyle \int_{r_0}^{r_{max}} u(0, r) dr = M $$ 
The problem becomes 
$$\partial_r( \tau(r, T^{\infty}) u^{\infty} - D \partial_r u^{\infty}) = 0$$
where the value $T^{\infty}$ depends on $u^{\infty}$
$$T^{\infty} = T_{g0} - \displaystyle \int_{r_0}^{r_{max}} \dfrac{V(r) - V_0}{V_l} u^{\infty}(r) \dfrac{4\pi r^2}{V_l} dr$$

So $\tau(r, T^{\infty}) u^{\infty}(r) - D \partial_r u^{\infty}(r) = K$, $K$ is constant.
with the boundary conditions it leads to $K = 0$, so $\tau(r, T^{\infty}) u^{\infty}(r) = D \partial_r u^{\infty}$

When $D \neq 0$ we have $u^{\infty}(r) = \displaystyle C \exp \left(\displaystyle  \int_{r_{0}}^{r} \dfrac{1}{D} \tau(s, T^{\infty}) ds \right)$.
The total number of cells is constant, $\displaystyle \int_{r_{0}}^{r_{max}} u^{\infty}(s) ds = M$, so we can characterize $C$ : 
$$ C = \displaystyle \dfrac{M}{\displaystyle \int_{r_{0}}^{r_{max}} \exp  \left( \int_{r_{0}}^{r} \dfrac{1}{D} \tau(s, T^{\infty}) ds \right) dr}.$$

We then have a system of 2 equations to solve in order to obtain $T^{\infty}$ and $u^{\infty}$: 
$$T^{\infty} = T_{g0} - \displaystyle  \int_{r_0}^{r_{max}} \dfrac{V(r) - V_0}{V_l} u^{\infty}(r) \dfrac{4\pi r^2}{V_l} dr$$
$$u^{\infty}(r) = \displaystyle \dfrac{M}{\displaystyle \int_{r_{0}}^{r_{max}} \exp  \left(\displaystyle  \int_{r_{0}}^{r} \dfrac{1}{D} \tau(s, T^{\infty}) ds \right) dr}  \, \exp \left(\displaystyle  \int_{r_{0}}^{r} \dfrac{1}{D} \tau(s, T^{\infty}) ds \right).$$


Can we compute the explicit integral of $\tau$ ? 
$$\begin{array}{rcl}
\dint_{r_{0}}^{r} \dfrac{1}{D} \tau(s, T) ds & = &  \dfrac{1}{D} \dint_{r_{0}}^{r} \left[\dfrac{aV_l}{4\pi}  \dfrac{T}{T + T_{\theta}} \dfrac{r_{\theta}^{n_r}}{s^{n_r}+ r_{\theta}^{n_r}} - V_l \dfrac{(B + bs^2)}{4\pi s^2} \dfrac{V(s) - V_0}{V(s) - V_0 + V_l  l_{\theta}}\right] ds \\
 &&\\
& = & \dfrac{1}{D} \dfrac{aV_l}{4\pi}  \dfrac{T}{T + T_{\theta}} \dint_{r_{0}}^{r} \dfrac{r_{\theta}^{n_r}}{s^{n_r}+ r_{\theta}^{n_r}} ds 

+ \dfrac{-1}{D} \dfrac{V_l}{4\pi} \dint_{r_{0}}^{r} \dfrac{\frac{4}{3} \pi B s^3 + \frac{4}{3} \pi b s^5 - V_0 b s^2 - V_0 B}{\frac{4}{3} \pi s^5 + (V_l l_{\theta} - V_0)s^2} ds \\
\end{array}
$$ 
Formal formula, from Wolframalpha, with $n_r = 3$, $k = r_{\theta}$ and $a_1 = \frac{4}{3} \pi b$, $a_2 = \frac{4}{3} \pi B$, $a_3 = - V_0 b$, $a_4 = - V_0 B$, $b_1 = \frac{4}{3} \pi$ and $b_2 = (V_l l_{\theta} - V_0)$ : 
$$\begin{array}{rcl}
\displaystyle \int \frac{k^3}{(x^3 + k^3)} dx & = & \dfrac{-1}{6} k \ln(x^2 - k x + k^2) \\
& & + \dfrac{1}{3} k \ln(k + x) \\
& & + \dfrac{1}{3} \sqrt{3} k \arctan((2 x - k)/(\sqrt{3} k))  \\
&& + constant
\end{array}$$

$$\begin{array}{l}
\displaystyle \int \frac{a_1 x^5 + a_2 x^3 + a_3 x^2 + a_4}{b_1 x^5 + b_2 x^2} dx = \\

\dfrac{1}{(6 b_1^{4/3} b_2^{4/3} x)} \times  \\

\left[-x \ln\left(b_1^{2/3} x^2 - b_1^{1/3} b_2^{1/3} x + b_2^{2/3}\right) \, (-b_2 (a_1 b_2^{2/3} + a_2 b_1^{2/3}) + a_3 b_1 b_2^{2/3} + a_4 b_1^{5/3}) \right. \\

 + 2 x \ln(b_1^{1/3} x + b_2^{1/3}) (-b_2 (a_1 b_2^{2/3} + a_2 b_1^{2/3}) + a_3 b_1 b_2^{2/3} + a_4 b_1^{5/3})  \\

 + 2 \sqrt{3} x \arctan(\frac{1}{\sqrt 3} (1 - \frac{2 b_1^{1/3} x}{b_2^{1/3}} )) (a_1 b_2^{5/3} - a_2 b_1^{2/3} b_2 - a_3 b_1 b_2^{2/3} + a_4 b_1^{5/3}) \\

 +\left. 6 a_1 b_1^{1/3} b_2^{4/3} x^2 - 6 a_4 b_1^{4/3} b_2^{1/3} \right] \\

 + constant \\
\end{array}$$



\end{document}