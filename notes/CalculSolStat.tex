\documentclass[11pt,a4paper]{article}
\usepackage[latin1,utf8]{inputenc}
\usepackage{natbib}
\usepackage[english, french]{babel}
\usepackage{amsmath}
\usepackage{amsfonts}
\usepackage{amssymb}
\usepackage{graphicx}
\usepackage{colortbl}
\usepackage{multirow}
\usepackage{xcolor}
\usepackage{geometry}
\usepackage{subfig}
\usepackage{tikz}
\maxdeadcycles=1000
%\geometry{ left = 60pt , right = 60pt }
\graphicspath{{./Fig/}}

\definecolor{mypurple}{RGB}{127,0,255}
\newcommand{\CA}[1]{{\color{mypurple} \emph{CA : #1} }}

\begin{document}
\begin{center}
\large \textbf{Stationary solution(s)}
\end{center}
\CA{update may 2020}

Search for stationary solution of our first model (see \textit{FirstModelAssumptions.pdf}), $u^{\infty}(r)$, that verifies $\partial_t u^{\infty}(t, r) = 0$ so the following equation
$$\partial_t u(t, r) + \partial_r( \tau(r, T(t))\, u(t, r) - D \partial_r u(t,r)) = 0$$
$$T(t) = T_{g0} - \int_{r_0}^{r_{max}} \dfrac{V(r) - V_0}{Vl} u(t, r) \dfrac{4\pi r^2}{Vl} dr$$
with $V(r) = \dfrac{4}{3} \pi r^3$ and where $\tau(r, T)$ is defined by 
$$\tau(r, T) = \dfrac{aV_l}{4\pi}  \dfrac{T}{T + T_{\theta}} \dfrac{r^{n_r}}{r^{n_r}+ r_{\theta}^{n_r}} - V_l \dfrac{(B + br^2)}{4\pi r^2} \dfrac{V(r) - V_0}{V(r) - V_0 + V_l  l_{\theta}}.$$
with the boundary conditions:
$$(\tau(r, T) u(t,r) - D \partial_r u(t,r)) |_{r_{0}} = 0$$
$$(\tau(r, T) u(t,r) - D \partial_r u(t,r)) |_{r_{max}} = 0$$
A consequence is that number of cells is constant, for all $t$, $$ \displaystyle \int_{r_0}^{r_{max}} u(t, r) dr = \displaystyle \int_{r_0}^{r_{max}} u(0, r) dr = M $$ 
The problem becomes 
$$\partial_r( \tau(r, T^{\infty}) u^{\infty} - D \partial_r u^{\infty}) = 0$$
where the value $T^{\infty}$ depends on $u^{\infty}$
$$T^{\infty} = T_{g0} - \displaystyle \int_{r_0}^{r_{max}} \dfrac{V(r) - V_0}{Vl} u^{\infty}(r) \dfrac{4\pi r^2}{Vl} dr$$

So $\tau(r, T^{\infty}) u^{\infty}(r) - D \partial_r u^{\infty}(r) = K$, $K$ is constant.
with the boundary conditions it leads to $K = 0$, so $\tau(r, T^{\infty}) u^{\infty}(r) = D \partial_r u^{\infty}$

When $D \neq 0$ we have $u^{\infty}(r) = \displaystyle C \exp \left(\displaystyle  \int_{r_{0}}^{r} \dfrac{1}{D} \tau(s, T^{\infty}) ds \right)$.
The total number of cells is constant, $\displaystyle \int_{r_{0}}^{r_{max}} u^{\infty}(s) ds = M$, so we can characterize $C$ : 
$$ C = \displaystyle \dfrac{M}{\displaystyle \int_{r_{0}}^{r_{max}} \exp  \left( \int_{r_{0}}^{r} \dfrac{1}{D} \tau(s, T^{\infty}) ds \right) dr}.$$

We then have a system of 2 equations to solve in order to obtain $T^{\infty}$ and $u^{\infty}$: 
$$T^{\infty} = T_{g0} - \displaystyle  \int_{r_0}^{r_{max}} \dfrac{V(r) - V_0}{Vl} u^{\infty}(r) \dfrac{4\pi r^2}{Vl} dr$$
$$u^{\infty}(r) = \displaystyle \dfrac{M}{\displaystyle \int_{r_{0}}^{r_{max}} \exp  \left(\displaystyle  \int_{r_{0}}^{r} \dfrac{1}{D} \tau(s, T^{\infty}) ds \right) dr}  \, \exp \left(\displaystyle  \int_{r_{0}}^{r} \dfrac{1}{D} \tau(s, T^{\infty}) ds \right).$$



\end{document}