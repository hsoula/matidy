\documentclass[11pt,a4paper]{article}
\usepackage[latin1,utf8]{inputenc}
\usepackage{natbib}
\usepackage[english, french]{babel}
\usepackage{amsmath}
\usepackage{amsfonts}
\usepackage{amssymb}
\usepackage{graphicx}
\usepackage{colortbl}
\usepackage{multirow}
\usepackage{xcolor}
\usepackage{geometry}
\usepackage{ulem}
\usepackage{subfig}
\usepackage{tikz}
\maxdeadcycles=1000
%\geometry{ left = 60pt , right = 60pt }
\graphicspath{{./Fig/}}

\definecolor{mypurple}{RGB}{127,0,255}
\newcommand{\CA}[1]{{\color{mypurple} \emph{CA : #1} }}

\begin{document}
\CA{update may 2020}

\paragraph{First model considered for adipocytes ``growing''.\\}
$T(t)$ Triglycerides ( = lipids) in the blood\\
$u(t,r)$ adipocytes \sout{number} density at time $t$ with respect to radius $r$\\
The link between the radius of a cell $r$ and $l$ the lipid amount in this cell is the following,
$$V(r) = \dfrac{4}{3} \pi r^3 \quad V_l l = V(r) - V_{0}$$
with $V_{0} = \dfrac{4}{3} \pi r_{0}^3$, minimum volume of an adipocyte, so $\dfrac{dl}{dt} = \dfrac{4\pi}{V_l} \dfrac{dr}{dt} r^2$.\\

Model the lipogenesis and the lipolysis (fluxes of triglycerides) : 
$$\dfrac{dl}{dt} = V_l a r^2 \dfrac{T}{T + T_{\theta}} \dfrac{r_{\theta}^{n_r}}{r^{n_r}+r_{\theta}^{n_r}} - V_l (B + br^2) \dfrac{l}{l + V_l l_{\theta}}$$
we rewrite with the variable $r$ : 
$$\dfrac{dr}{dt} = \dfrac{aV_l}{4\pi}  \dfrac{T}{T + T_{\theta}} \dfrac{r^{n_r}}{r^{n_r}+ r_{\theta}^{n_r}} - V_l \dfrac{(B + br^2)}{4\pi r^2} \dfrac{V(r) - V_0}{V(r) - V_0 + V_l  l_{\theta}} = \tau(r, T)$$
From these fluxes, the dynamics of the \sout{number} density of adypocytes $u$ is described by 
$$\partial_t u(t, r) + \partial_r( \tau(r, T(t)) u(t,r) - D \partial_r u(t,r)) = 0$$
with $D$ a diffusion coefficient in $\mu m^2 . h^{-1}$.

The intracellular quantity of triglycerides (in $nmol$) is $U(t) = \displaystyle \int l u dl$. Changing to variable $r$, 
$$U(t) = \int_{r_0}^{r_{max}} \dfrac{V(r) - V_0}{Vl} u(t, r) \dfrac{4\pi r^2}{Vl} dr$$

The total amount of triglycerides is assumed constant over time : $\dfrac{d}{dt}(T(t) + U(t)) = 0$, we denote the initial quantity of triglycerides by $T_{g0}$ so for all time $t$, $$T(t) = T_{g0} - U(t) =  T_{g0} - \int_{r_0}^{r_{max}} \dfrac{V(r) - V_0}{Vl} u(t, r) \dfrac{4\pi r^2}{Vl} dr$$
Table \ref{Tab:VarParUnits} sums up all parameters and variable, including units.

\begin{table}
\begin{tabular}{l|l|l}
name & unit & description \\
\hline 
$t$ & $h$ & time in hours \\
$r$ & $\mu m$ & adipocyte radius \\
$T(t)$ &  $nnol$ & triglycerides ( = lipids) in the blood \\
$u(t,r)$ & - & adipocyte density at time $t$ with respect to $r$ \\
$l$ & $nnol$ & the lipid amount in an adipocyte \\
$V_0$ & $\mu m ^3$ & minimum volume of an adipocyte : empty of lipid \\
$r_0$ & $\mu m$ & minimum adipocyte radius, when it contains no lipid \\
$Vl$ &  $\mu m ^3 . nmol^{-1}$ & volume taken by 1 $nmol$ of triglyceride \\ 
$a$ & $nmol. \mu m ^{-2} . h^{-1}$ & the surface limited diffusion rate  \\
$T_{\theta}$ & $nmol$ & constant of Michaelis-Menten term $T/(T+T_{\theta}) = 1/2$ when $T = T_{\theta}$ \\
$r_{\theta}$ & $\mu m$ & Hill function cutoff radius, for $r > r_{\theta}$ lipogenesis rate decreases dramatically \\
$n_r$ & - & $n_r > 0$ power in the Hill function \\
$B$ & $nmol . h^{-1}$ & basal lipolysis \\
$b$ & $nmol .\mu m ^{-2} . h^{-1}$ & part of lipolysis depending on organic compounds (catecholamines) \\
$l_{\theta}$ & $nmol$ & constant in the term when cell lipid content is low \\
$D$ & $\mu m^2 . h^{-1}$ & diffusion coefficient represent size fluctuation \\
$U(t)$ & $nmol$ & intracellular quantity of triglycerides \\
$T_{g0}$ & $nmol$ & total amount of triglycerides initially (and assumed constant)
\end{tabular}\caption{sum up all variables and parameters, including units} \label{Tab:VarParUnits}
\end{table}

\paragraph{boundary/initial conditions.}

$u(0, r) =$ Gaussian density (minimum = $r_{0}$), first test : from unimodal initial condition can we recover the bimodal distribution which is observed \\
$T(t) = T_{g0} - U(t)$\\
$(\tau(r, T) u(t,r) - D \partial_r u(t,r)) |_{r_{max}} = 0$

\subparagraph{Recruitment of new adipocytes.} 

First, we assume a constant number of cells : $(\tau(r, T) u(t,r) - D \partial_r u(t,r)) |_{r_{0}}= 0$


Then, we will assumed that when the level of triglycerides increases too largely, pre-adipocytes differentiate into adipocytes. This recruitment is modeled with the $r_0$ BC, as follows,
$$(\tau(r, T) u(t,r) - D \partial_r u(t,r)) |_{r_{0}}= f(T(t))$$ with $f(T(t)) = \alpha T(t)$ or $f(T(t)) = \alpha \dfrac{T(t)}{(T(t)+\kappa_{\theta})}$ 

\end{document}